\section{Related Work} \label{sec:related}

The Algorithm Selection Problem~\cite{rice1976algorithm} consists in finding a
\emph{mapping} between \emph{problems} and \emph{algorithms} that minimizes the
time to solve all instances in a problem set. The algorithms that compose a set
can represent different abstractions, such as programs, heuristics, or
configurations. The set of problems usually contains instances of a problem.
Bougeret \emph{et al.}~\cite{bougeret2009combining} proved that the Algorithm
Selection Problem is NP-complete when calculating static distributions of
algorithms in parallel machines.  Guo~\cite{guo2003algorithm} proved the
problem is undecidable in the general case.

Rice's conceptual framework formed the foundation of autotuners in various
problem domains.  In 1997, the PHiPAC system~\cite{bilmes1997phipac} used code
generators and search scripts to automatically generate high performance code
for matrix multiplication. Since then, systems tackled different domains with a
diversity of strategies. Whaley \emph{et al.}~\cite{whaley1998atlas} introduced
the ATLAS project, that optimizes dense matrix multiply routines. The
OSKI~\cite{vuduc2005oski} library provides automatically tuned kernels for
sparse matrices. The FFTW~\cite{frigo1998fftw} library provides tuned C
subroutines for computing the Discrete Fourier Transform.  In an effort to
provide a common representation of multiple parallel programming models, the
INSIEME compiler project~\cite{jordan2012multi} implements abstractions for
OpenMP, MPI and OpenCL, and generates optimized parallel code for heterogeneous
multi-core architectures.

Some autotuning systems provide generic tools that enable the implementation of
autotuners in various domains. PetaBricks~\cite{ansel2009petabricks} is a
language, compiler and autotuner that introduces abstractions, such as the
\lq\lq{}\emph{either...or}\rq\rq{} construct, that enable programmers to define
multiple algorithms for the same problem.  The ParamILS
framework~\cite{hutter2009paramils} applies stochastic local search methods
for algorithm configuration and parameter tuning.  The OpenTuner
framework~\cite{ansel2014opentuner} provides ensembles of techniques that
search spaces of program configurations. Bosboom \emph{et al.} and Eliahu use
OpenTuner to implement a domain specific language for data-flow
programming~\cite{bosboom2014streamjit} and a framework for recursive parallel
algorithm optimization~\cite{eliahu2015frpa}.
