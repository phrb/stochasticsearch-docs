\section{Introduction} \label{sec:intro}

This paper introduces \emph{StochasticSearch.jl}, a package for the Julia
language~\cite{bezanson2012julia, bezanson2014julia} that provides building
blocks for the implementation of different stochastic local search
methods~\cite{hoos2004stochastic, hoos2015overview}. The package also provides
types for the definition of program search spaces, and enables the application
of multiple parallel instances of search techniques to program autotuning.

The program autotuning problem fits in the framework of the Algorithm Selection
Problem, introduced by Rice in 1976~\cite{rice1976algorithm}. The objective of
an autotuner is to select the best algorithm, or algorithm configuration, for
each instance of a problem.  Algorithms or configurations are selected by their
performance, which is measured by the time to solve the problem instance, the
accuracy of the solution or the energy consumed.

The set of all possible algorithms and configurations that solve a problem
define a \emph{search space}. Guided by the performance metrics, various
optimization techniques search this space for the algorithm or configuration
that best solves the problem.

Autotuners can specialize in domains such as matrix
multiplication~\cite{bilmes1997phipac}, dense~\cite{whaley1998atlas} or
sparse~\cite{vuduc2005oski} matrix linear algebra, and parallel
programming~\cite{jordan2012multi}. Other autotuning frameworks provide more
general tools for the representation and search of program configurations,
enabling the implementation of autotuners for different problem
domains~\cite{ansel2014opentuner, hutter2009paramils}.

The \emph{StochasticSearch.jl} package casts the Algorithm Selection Problem as
a search problem~\cite{ansel2014opentuner, ansel2014phd}. The user represents
the program to be tuned as set of parameters. Each point in the space defined
by this representation generates a different program, whose performance is
measured. The measurements are used to guide stochastic local search techniques
that produce new points.  Multiple instances of different search techniques can
be executed in parallel, speeding up the rate at which performance improvements
are found.

% Highlight of results (to be produced) goes here!
The contribution of this paper is the implementation of the
\emph{StochasticSearch.jl} package, a framework for program autotuning that
provides building blocks for the implementation of stochastic local search
methods.  The remaining of the paper is organized as follows.
Section~\ref{sec:related} presents related work. Section~\ref{sec:arch}
describes the implementation of the package. Section~\ref{sec:experiments}
describes the problem domains, their representations and the experimental
settings. Section~\ref{sec:results} discusses the results.
Section~\ref{sec:conclusion} presents the conclusion.
