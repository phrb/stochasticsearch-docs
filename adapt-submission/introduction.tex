\section{Introduction}

This paper introduces \textit{StochasticSearch.jl}, a package for the Julia
language~\cite{bezanson2012julia, bezanson2014julia} that provides building
blocks for the implementation of different stochastic local search
methods~\cite{hoos2004stochastic, hoos2015overview}, such as simulated
annealing~\cite{kirkpatrick1983optimization}. The package also provides types
for the definition of program search spaces, and allows the application of
multiple parallel instances of search techniques to program autotuning.

Program autotuning is a solution to the Algorithm Selection Problem, first
published by Rice in 1976~\cite{rice1976algorithm}. Autotuning systems usually
model features of computer architectures and algorithms in a certain problem
domain, defining a \emph{search space}. Optimization techniques specific for a
problem domain are used to search this space for the algorithm or configuration
that best solves the problem. Different metrics can orient the search, such as
the time to solve the problem, the accuracy of the solution or the energy
consumed.  

Autotuning systems, or autotuners, can be specific to domains such as matrix
multiplication~\cite{bilmes1997phipac} and dense or sparse matrix linear
algebra~\cite{whaley1998atlas, vuduc2005oski}.  Autotuning frameworks such as
OpenTuner~\cite{ansel2014opentuner} and ParamILS~\cite{hutter2009paramils}
implement domain-agnostic tools for the representation and search of program
configurations.

The StochasticSearch.jl package casts the Algorithm Selection Problem as a
search problem~\cite{ansel2014opentuner, ansel2014phd}. The user represents the
program to be tuned as set of parameters. Each point in the space defined by
this set generates a different program, whose performance is measured. The
measurements are used to guide stochastic local search techniques that produce
new points.  Multiple instances of different search techniques
can be executed in parallel, speeding up the rate at which performance
improvents are found.
